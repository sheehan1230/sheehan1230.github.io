\begin{abstract}
Vision-language models (VLMs) have shown remarkable capabilities in integrating linguistic and visual reasoning but remain fundamentally limited in understanding dynamic spatiotemporal interactions. Humans effortlessly track and reason about object movements, rotations, and perspective shifts—abilities essential for robust dynamic real-world understanding yet notably lacking in current VLMs. In this paper, we introduce \texttt{VLM4D}, the first benchmark specifically designed to evaluate the spatiotemporal reasoning capabilities of VLMs. Our benchmark comprises diverse real-world and synthetic videos accompanied by carefully curated question-answer pairs emphasizing translational and rotational motions, perspective awareness, and motion continuity. Through comprehensive evaluations of state-of-the-art open and closed-source VLMs, we identify significant performance gaps compared to human baselines, highlighting fundamental deficiencies in existing models. Extensive analysis reveals that VLMs struggle particularly with integrating multiple visual cues and maintaining temporal coherence. We further explore promising directions, such as leveraging 4D feature field reconstruction and targeted spatiotemporal supervised fine-tuning, demonstrating their effectiveness in enhancing spatiotemporal comprehension. Our work aims to encourage deeper exploration into improving VLMs’ spatial and temporal grounding, paving the way towards more capable and reliable visual intelligence for dynamic environments.
\end{abstract}